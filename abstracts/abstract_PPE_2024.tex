\noindent Economists have long acknowledged the centrality of innovation in economic growth and of institutions, defined as the “rules of the game”, as a determinant of innovation outcomes. This paper studies the impact on innovation of a special case of institutional change, namely the attainment of national independence. Historical episodes of this type, such as those from nineteenth-century Europe, are complex shocks that may change several types of formal institutions, such as legal codes and patent systems, as well as the general socio-economic context economic agents face, e.g. market access and cultural homogeneity, thereby affecting a country's innovative performance.

\noindent We focus on one important historical episode, namely Italy’s unification. The richest land of the Western world until the end of the Middle Ages, the politically divided peninsula underwent decline in the modern era, and large parts of it were ruled by foreign powers (France, Spain, and Austria) until the 19th century, when the country’s Risorgimento led to unification under a domestic leadership. We leverage quasi-random variation in the assignment of independence from foreign rule in the former Lombardo-Venetian Kingdom in 1859 to investigate its impact on innovation and economic development in a regression discontinuity design.

\noindent We measure innovation using a plurality of gauges, namely data from the universal exhibitions taking place in Paris (1855, 1867, 1878, 1889, and 1900), as well as patent data from Italy and Austria. Exhibition data also allow evaluating changes in the economic complexity of locations’ productions. Our preliminary results show a negative effect of national independence on innovation in the short run.
