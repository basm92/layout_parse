\documentclass[12pt]{extarticle}
\usepackage[utf8]{inputenc}
\usepackage[margin=1in]{geometry}

\usepackage{placeins}   % Fot \FloatBarrier

\usepackage[cmintegrals,cmbraces]{newtxmath}
%\usepackage{ebgaramond}

\usepackage{amsmath}
\usepackage{amsfonts}
%\usepackage{amssymb}

%table numbers within section
\usepackage{chngcntr}
\counterwithin{table}{section}

\usepackage{natbib}
\usepackage{graphicx}
\usepackage{booktabs}
\usepackage{longtable}
\usepackage{caption}

%\frenchspacing
\usepackage{setspace}
\setstretch{1.1}

\usepackage{graphicx}
\usepackage{url}
\usepackage[colorlinks=true, urlcolor = orange, linkcolor=blue, citecolor=purple]{hyperref}

\usepackage{xcolor}
%\usepackage{sectsty}
\usepackage{appendix}

%\subsectionfont{\color{cyan!70!blue}}  % sets colour of chapters
%\sectionfont{\color{blue!60!black}}  % sets colour of sections

%Tables:
\usepackage{dcolumn}
\usepackage{float}
\usepackage{caption}
\usepackage{tabularx}

% Landscape mode
\usepackage{lscape}

% Modelsummary
\usepackage{threeparttable}
\usepackage{booktabs}
\usepackage{siunitx}
%\newcolumntype{d}{S[input-symbols = ()]}
\newcolumntype{d}{S[
    input-open-uncertainty=,
    input-close-uncertainty=,
    parse-numbers = false,
    table-align-text-pre=false,
    table-align-text-post=false
 ]}

%% Regression tables
\usepackage{tabularray}
\usepackage{float}
\usepackage{graphicx}
\usepackage{rotating}
\usepackage[normalem]{ulem}
\UseTblrLibrary{booktabs}
\UseTblrLibrary{siunitx}
\newcommand{\tinytableTabularrayUnderline}[1]{\underline{#1}}
\newcommand{\tinytableTabularrayStrikeout}[1]{\sout{#1}}
\NewTableCommand{\tinytableDefineColor}[3]{\definecolor{#1}{#2}{#3}}

\title{Unification and Innovation\footnote{We are grateful to XXXX for insightful comments and suggestions.}}
\author{Giacomo Domini \and Bas Machielsen\footnote{Both Utrecht University School of Economics, Utrecht University, Kriekenpitplein 21-22, 3584 EC Utrecht, the Netherlands; e--mail: \href{mailto:a.h.machielsen@uu.nl}{a.h.machielsen@uu.nl}}}
\date{\today}

\begin{document}

\maketitle

\begin{center}
    For the latest version, please \href{http://link.com/paper.pdf}{click here.}
\end{center}

\begin{center} \textbf{Abstract:} \end{center}

%\noindent Under Austro-Hungarian rule, the Lombardy and Veneto regions were subject to centralized, autocratic administration. After annexation by the Kingdom of Italy, they became part of a constitutional monarchy, where citizens had greater political participation and representation. We exploit natural experiments arising from the reunification of Lombardy (1859) and Venetia (1866) with the rest of Italy to study the impact of the accompanying democratization and national reintegration on economic activity through the innovation channel. We combine a spatial regression discontinuity-design with panel data on world industrial expositions to study the impact of reunification on product complexity and innovative activity at the city-level, comparing increases in innovative activity of towns that remained in the Austro-Hungarian empire with towns that were incorporated in Italy. Our results show that [XXX]. 

\noindent Economists have long acknowledged the centrality of innovation in economic growth and of institutions, defined as the “rules of the game”, as a determinant of innovation outcomes. This paper studies the impact on innovation of a special case of institutional change, namely the attainment of national independence. Historical episodes of this type, such as those from nineteenth-century Europe, are complex shocks that may change several types of formal institutions, such as legal codes and patent systems, as well as the general socio-economic context economic agents face, e.g. market access and cultural homogeneity, thereby affecting a country's innovative performance.

\noindent We focus on one important historical episode, namely Italy’s unification. The richest land of the Western world until the end of the Middle Ages, the politically divided peninsula underwent decline in the modern era, and large parts of it were ruled by foreign powers (France, Spain, and Austria) until the 19th century, when the country’s Risorgimento led to unification under a domestic leadership. We leverage quasi-random variation in the assignment of independence from foreign rule in the former Lombardo-Venetian Kingdom in 1859 to investigate its impact on innovation and economic development in a regression discontinuity design.

\noindent We measure innovation using a plurality of gauges, namely data from the universal exhibitions taking place in Paris (1855, 1867, 1878, 1889, and 1900), as well as patent data from Italy and Austria. Exhibition data also allow evaluating changes in the economic complexity of locations’ productions. Our preliminary results show a negative effect of national independence on innovation in the short run.



\textbf{JEL Classifications:} N14, D72, H71

\clearpage

\section{Introduction}

% Complementary framing: periphery vs. central part 
The effect of democratization on economic activity has been debated in many disciplines. Despite some evidence to the contrary, the consensus now seems to be that there likely is a positive influence of democracy on economic activity. Various empirical studies have shown that, in a cross-country context, democratization causally influence economic growth. Theoretically, there are many possible channels through which democratization can lead to economic growth. % Mention some stuff

Empirically, however, there is a lack of evidence on the specific mechanisms at work. In this paper, we propose to investigate one specific mechanism: economic growth through innovative activity. To investigate this channel, we look at the case of Italian unification, involving the successive integration of the Lombardy (1859) and Veneto (1866) regions into the Kingdom of Italy. Under Austrian rule, Lombardy and Veneto were subject to a centralized, autocratic administration. After annexation by the Kingdom of Italy, they became part of a constitutional monarchy, where its citizens had greater political participation and representation. Local administrations and governance structures were overhauled to align with the new Italian government, which introduced a more centralized and unified system.

Additionally, the Austrian Empire had implemented a policy of Germanization in some areas of Lombardy and Veneto, which affected local language and culture, and with Italian unification, there was a renewed emphasis on the Italian language and culture. The legal system also underwent many changes: after incorporation, the legal and judicial systems in both Lombardy and Veneto shifted from Austrian law to Italian law. This change included the adoption of the Italian legal code and legal practices.
% Make these legal practices and patent law specifically more precise

Finally, the Italian government explicitly aimed to modernize and develop the economy of these regions. Investments in infrastructure, transportation, and industry were made to promote economic growth and integration into the wider Italian economy. The adoption of the Italian lira as the official currency replaced the Austrian currency.

We leverage two arguably exogenous border changes involving the incorporation of Lombardy (1859) and Veneto (1866) into Italy to identify the influence of all aforementioned impetuses on economic and innovative activity. Our first treatment, the incorporation of Lombardy, resulted from the termination of the Franco-Austrian War, or the \textit{Second War of Independence} (1859), when the Kingdom of Sardinia-Piedmont, the driving force behind Italian unification, allied with France and went to war against the Austrian Empire. The war ended with the Treaty of Villafranca, in which the French Emperor Napoleon III brokered peace between the warring parties, which led to the annexation of Lombardy, including its capital, Milan, by the Kingdom of Sardinia-Piedmont.

Our second treatment, the incorporation of Veneto, revolves around the termination of the \text{Third War of Independence}, in 1866. This war, waged in parallel to the larger Prussian--Austro-Hungarian war, was terminated by the peace of Prague, which stipulated the Austrian cession of the Veneto region. The Peace of Prague was followed up by the Austrian-Italian Treaty of Vienna, which confirmed the cession of the territory to Italy. However, the peace treaty stated that the annexation of Venetia and Mantua would have become effective only after a referendum, which was held on 21 and 22 October, and the result was an overwhelming success. 

% How did the war go and where was it fought? No confounding factor?

Our identification strategy leverages the randomness involved in the exact determination of the border and the comparability between cities located at similar distances to the new border. 

We also contribute to the democratization literature by providing a data-driven approach to democratization questions on the micro level. Most empirical evidence comes from variation at the country-year level, whereas this study explicitly focuses on the more granular city-level. 


\section{Data \& Methods}

\subsection{Exhibition Data} 
The data employed in this paper are retrieved from the official catalogues of the five Parisian universal exhibitions introduced above (Exposition des produits de l’industrie de toutes les nations, 1855, 1855; Exposition universelle de 1867 à Paris, 1867; Exposition universelle internationale de 1878 à Paris, 1878; Exposition universelle internationale de 1889, à Paris 1889a, Exposition universelle internationale de 1889, à Paris 1889b, Exposition universelle internationale de 1889, à Paris 1889c, Exposition universelle internationale de 1889, à Paris 1889d; Exposition internationale universelle de 1900 à Paris, 1900). This section discusses technical aspects concerning the original data classification, the construction of the database, and the variables computed from exhibition data (RSCA and ECI). More detailed information on these issues is provided in the appendix.

\subsection{Geographical Data}

\subsection{Data Pipeline}

% also incorporate with geographical data, we can include 0 observations to the dataset

% mention the packages I use in here 

Our pipeline starts from the digitized exposition data from an exposition at time $t$. Using a layout detection deep learning algorithm, we identify (i) the particular class of products being exhibited, and for each class, we then identify the observations represented by row entries. 

Observations are defined as meaning an exhibitor $g$ (identified by a name) from town $i$, exhibiting a product from class $j$ at time of the exhibition $t$. Initially, we end up with raw strings containing this information. We use a rule-based approach to extract the town $i$ from the string, and then match it using a pre-trained language model-based approach to a predetermined list of city names, allowing us to geo-code our observations according to the city of origin. 

Our baseline analyses involve aggregating this data to the city-class-year level $ijt$: we thus end up with a count variable for number of exhibitors in each class $j$ in exhibition year $t$. We augment this dataset by all towns in the affected areas that had no exhibitors in clas $j$ in exhibition year $t$. 


\subsection{Methods}

% think about: construct a complexity measure from the verbal descriptions of the product

% complexity is a exhibitor-region-class level variable, but innovative activity or something is a region-class level variable

Our point of reference encompasses a cross-sectional design around municipalities $i$:

\begin{equation*}
    Y_{i, 1867} = \alpha + \beta D_i + \epsilon_i 
\end{equation*}

where $D_i$ is a treatment indicator, equal to 1 if city $i$ has become part of Italy in 1859 and 1866 respectively. We might also consider a local regression discontinuity design, where $D_i$ is defined as \textit{distance to the border}, multiplied by -1 if the city is located in the Austro-Hungarian Empire in 1867. \footnote{We can compare for before-treatment similarity by running $Y_{i, 1855} = \alpha_0 + \alpha_1 D_i + u_i$}

A second approach might be leveraging the panel data from different expositions at times $t \in \{ 1855, 1867\}$. In that case, the model we can estimate for $Y$ in city $i$ at time $t$: Methods

\begin{equation*}
    Y_{it} = \alpha_i + \beta_1 \text{Post}_t + \beta_2 D_i \cdot \text{Post}_t + \epsilon_{it}\footnote{We could also replace the $\alpha_i$ by a "mean difference" $D_i$, making it a standard difference-in-difference. The model in the equation, is however more general.}
\end{equation*}

in which we control for level differences between cities $i$ by the $\alpha_i$'s, and we compare villages at the same distance from the border before and after the unification. Our coefficient of interest is $\beta_2$, the difference between the treatment and control groups after the treatment. $D_i$ 

In this regression, we could add distance to the border as a control variable, and potentially we can also weight the observations by the inverse distance to the border. Then, it comes closer to a geographical RDD design. 

Finally, we can also opt for an explicit geographical local regression discontinuity estimate combine with a difference in difference methodology: 

\begin{equation*}
    Y_{it} - Y_{it-1} = \alpha + \beta D_i + \epsilon_{it}
\end{equation*}

where in the geographical RDD design, we define $D_i$ as being the distance to the border. 


% Take into account, super-city level of aggregation if we want
% Also; city 

% Patent data in the future

\section{Results}


\subsection{RD Estimates of Unification on Innovative Activity}

In Table \ref{tab:rd_analysis_number}, the estimates of unification on innovations are reported. In this analysis, we look at the cross-sectional innovative activity in municipality $i$ and we find there is a discontinuous increase in innovative activity in municipalities that are part of the newly-annexed Veneto region compared to the earlier-annexed Lombardy region. 

\begin{center}
    [Table \ref{tab:rd_analysis_number} here]
\end{center}

\subsection{DiD Estimates of Unification on Innovative Activity}

In Table \ref{tab:did_analysis_number}, we analyze the influence of unification on innovative activity using a difference in difference strategy, comparing the innovations in 1855 with the innovations in 1867 in both region. The coefficient of interest is Year (1867) x Veneto, the estimate of the effect of reunification relative to the counterfactual increase in innovation in Lombardy. In this analysis, we find no discernable effect of reunification on the count of innovations. 

\begin{center}
    [Table \ref{tab:did_analysis_number} here]
\end{center}

\section{Mechanisms}

% Democratization
%% Almost equally undemocratic - Austro-Hungary vs. Italy
%% What kind of variation to use? 

% Homophily 
%% Ethnic and linguistic coherence (Oded Galor and descendants)
%% Changes in the market potential of different regions
%% Tariffs and Trade barriers? 
%% After becoming part of Italy, entire A-H Market became inaccessible
%%% Railway and Road Data in conjunction 
%%% For some regions this may be bad
%%% But for others not that bad because there were already huge transport costs

% In 1867
% Veneto joined in 1866
% Lombardo joined in 1859 

% Two exercises now: complexity measure on the left hand side
% Have a look at the austrian patent data and Scrape


% New file for Italy 1878, 1889 and 1900

\section{Conclusion}

\bibliographystyle{econ}
\bibliography{bibliography}

\clearpage

\appendix

\section{Tables Main Text}

\begin{table}[!h]

\caption{\label{tab:rd_analysis_number}Estimates of Italian Unification on Innovative Activity}
\centering
\resizebox{\linewidth}{!}{
\begin{threeparttable}
\begin{tabular}[t]{lllll}
\toprule
\multicolumn{1}{c}{ } & \multicolumn{2}{c}{Log Innovations} & \multicolumn{2}{c}{Ihs Innovations} \\
\cmidrule(l{3pt}r{3pt}){2-3} \cmidrule(l{3pt}r{3pt}){4-5}
\multicolumn{1}{c}{ } & \multicolumn{1}{c}{No FE} & \multicolumn{1}{c}{FE} & \multicolumn{1}{c}{No FE} & \multicolumn{1}{c}{FE} \\
\cmidrule(l{3pt}r{3pt}){2-2} \cmidrule(l{3pt}r{3pt}){3-3} \cmidrule(l{3pt}r{3pt}){4-4} \cmidrule(l{3pt}r{3pt}){5-5}
  & (1) & (2) & (3) & (4)\\
\midrule
Estimate & 0.140** & 0.149** & 0.175** & 0.189**\\
SE (BC) & (0.063) & (0.052) & (0.079) & (0.065)\\
Mean DV Treated 50km & 0.083 & 0.083 & 0.103 & 0.103\\
Mean DV Control 50km & 0.062 & 0.062 & 0.078 & 0.078\\
N (Treated) & 667 & 667 & 667 & 667\\
N (Control) & 1549 & 1549 & 1549 & 1549\\
Bandwidth & 41406.168 & 45338.659 & 42289.180 & 46458.038\\
\bottomrule
\end{tabular}
\begin{tablenotes}[para]
\item \textit{Note: } 
\item Table showing coefficient estimates and bias-corrected standard errors of the impact of Italian Unification on innovative activity. The dependent variable is log or ihs no. of innovations and the independent (running) variable is distance to the border. The bandwidth is estimated using the MSE-optimal bandwidth from \cite{cattaneo2019practical}. The estimates control for area, angle to border and innovation. Models (2) and (4) are conditional on province fixed effects. *: p < 0.10, **: p < 0.05, ***: p < 0.01.
\end{tablenotes}
\end{threeparttable}}
\end{table}

\clearpage

\begin{table}[!h]

\caption{\label{tab:did_analysis_number}Difference-in-difference Estimates of Unification on Innovation}
\centering
\fontsize{9}{11}\selectfont
\begin{threeparttable}
\begin{tabular}[t]{lcccc}
\toprule
\multicolumn{1}{c}{ } & \multicolumn{2}{c}{OLS} & \multicolumn{2}{c}{Poisson} \\
\cmidrule(l{3pt}r{3pt}){2-3} \cmidrule(l{3pt}r{3pt}){4-5}
  & (1) & (2) & (3) & (4)\\
\midrule
Year (1867) & \num{0.037}*** & \num{0.037}*** & \num{1.345}*** & \num{1.345}***\\
 & (\num{0.006}) & (\num{0.006}) & (\num{0.225}) & (\num{0.226})\\
Veneto & \num{-0.029} & \num{0.063} & \num{-1.248} & \num{1.111}\\
 & (\num{0.021}) & (\num{0.039}) & (\num{0.873}) & (\num{0.741})\\
Year (1867) x Veneto & \num{0.007} & \num{0.007} & \num{-0.130} & \num{-0.130}\\
 & (\num{0.013}) & (\num{0.013}) & (\num{0.367}) & (\num{0.368})\\
Elevation & \num{0.000}*** & \num{0.000}*** & \num{-0.003}*** & \num{-0.004}***\\
 & (\num{0.000}) & (\num{0.000}) & (\num{0.001}) & (\num{0.001})\\
Longitude & \num{-0.021}* & \num{-0.049}** & \num{0.111} & \num{-0.735}\\
 & (\num{0.011}) & (\num{0.022}) & (\num{0.219}) & (\num{0.622})\\
Latitude & \num{0.081}*** & \num{0.083}** & \num{2.126}*** & \num{2.756}***\\
 & (\num{0.026}) & (\num{0.036}) & (\num{0.713}) & (\num{0.979})\\
Area & \num{0.004}*** & \num{0.004}*** & \num{0.021}*** & \num{0.040}***\\
 & (\num{0.001}) & (\num{0.001}) & (\num{0.003}) & (\num{0.005})\\
Angle to Border & \num{0.000} & \num{0.000} & \num{-0.005}* & \num{-0.003}\\
 & (\num{0.000}) & (\num{0.000}) & (\num{0.002}) & (\num{0.002})\\
\midrule
N & \num{4432} & \num{4432} & \num{4432} & \num{4432}\\
Province FE & No & Yes & No & Yes\\
\bottomrule
\multicolumn{5}{l}{\rule{0pt}{1em}* p $<$ 0.1, ** p $<$ 0.05, *** p $<$ 0.01}\\
\end{tabular}
\begin{tablenotes}[para]
\item \textit{Note: } 
\item Dependent variables: Number of innovations in municipality $i$. The coefficient of interest is the Year x Group{Veneto} variable. The control variables are latitude, longitude, elevation, and the analysis is conditional on provinde fixed-effects. Standard errors are clustered at the municipality level.
\end{tablenotes}
\end{threeparttable}
\end{table}

\clearpage

\begin{table}[!h]

\caption{\label{tab:rd_analysis_placebo}Placebo Test of Italian Unification on Innovative Activity}
\centering
\resizebox{\linewidth}{!}{
\begin{threeparttable}
\begin{tabular}[t]{lllll}
\toprule
\multicolumn{1}{c}{ } & \multicolumn{2}{c}{Log Innovations} & \multicolumn{2}{c}{Ihs Innovations} \\
\cmidrule(l{3pt}r{3pt}){2-3} \cmidrule(l{3pt}r{3pt}){4-5}
\multicolumn{1}{c}{ } & \multicolumn{1}{c}{No FE} & \multicolumn{1}{c}{FE} & \multicolumn{1}{c}{No FE} & \multicolumn{1}{c}{FE} \\
\cmidrule(l{3pt}r{3pt}){2-2} \cmidrule(l{3pt}r{3pt}){3-3} \cmidrule(l{3pt}r{3pt}){4-4} \cmidrule(l{3pt}r{3pt}){5-5}
  & (1) & (2) & (3) & (4)\\
\midrule
Estimate & -0.025* & -0.020 & -0.035* & -0.023\\
SE (BC) & (0.012) & (0.013) & (0.017) & (0.015)\\
Mean DV Treated 50km & 0.015 & 0.015 & 0.018 & 0.018\\
Mean DV Control 50km & 0.007 & 0.007 & 0.009 & 0.009\\
N (Treated) & 667 & 667 & 667 & 667\\
N (Control) & 1549 & 1549 & 1549 & 1549\\
Bandwidth & 23619.409 & 18088.229 & 24106.902 & 18507.961\\
\bottomrule
\end{tabular}
\begin{tablenotes}[para]
\item \textit{Note: } 
\item Table showing coefficient estimates and bias-corrected standard errors of a placebo test, studying the impact of Italian Unification on innovative activity before unification took place. The dependent variable is log or ihs no. of innovations and the independent (running) variable is distance to the border. The bandwidth is estimated using the MSE-optimal bandwidth from \cite{cattaneo2019practical}. The estimates control for area, angle to border and innovation. Models (2) and (4) are conditional on province fixed effects. *: p < 0.10, **: p < 0.05, ***: p < 0.01.
\end{tablenotes}
\end{threeparttable}}
\end{table}

\clearpage

\end{document}