\documentclass[12pt]{extarticle}
\usepackage[utf8]{inputenc}
\usepackage[margin=1in]{geometry}

\usepackage[cmintegrals,cmbraces]{newtxmath}
%\usepackage{ebgaramond}

\usepackage{amsmath}
\usepackage{amsfonts}
%\usepackage{amssymb}

%table numbers within section
\usepackage{chngcntr}
\counterwithin{table}{section}

\usepackage{natbib}
\usepackage{graphicx}
\usepackage{booktabs}
\usepackage{longtable}
\usepackage{caption}

%\frenchspacing
\usepackage{setspace}
\setstretch{1.1}

\usepackage{graphicx}
\usepackage{url}
\usepackage[colorlinks=true, urlcolor = orange, linkcolor=blue, citecolor=purple]{hyperref}

\usepackage{xcolor}
%\usepackage{sectsty}
\usepackage{appendix}

%\subsectionfont{\color{cyan!70!blue}}  % sets colour of chapters
%\sectionfont{\color{blue!60!black}}  % sets colour of sections

%Tables:
\usepackage{dcolumn}
\usepackage{float}
\usepackage{caption}
\usepackage{tabularx}

% Landscape mode
\usepackage{lscape}

% Modelsummary
\usepackage{threeparttable}
\usepackage{booktabs}
\usepackage{siunitx}
%\newcolumntype{d}{S[input-symbols = ()]}
\newcolumntype{d}{S[
    input-open-uncertainty=,
    input-close-uncertainty=,
    parse-numbers = false,
    table-align-text-pre=false,
    table-align-text-post=false
 ]}

\title{Reunification and Innovation\footnote{We are grateful to XXXX for insightful comments and suggestions.}}
\author{Giacomo Domini \and Bas Machielsen\footnote{Both Utrecht University School of Economics, Utrecht University, Kriekenpitplein 21-22, 3584 EC Utrecht, the Netherlands; e--mail: \href{mailto:a.h.machielsen@uu.nl}{a.h.machielsen@uu.nl}}}
\date{\today}

\begin{document}

\maketitle

\begin{center}
    For the latest version, please \href{http://link.com/paper.pdf}{click here.}
\end{center}

\begin{center} \textbf{Abstract:} \end{center}

\noindent There is ample evidence in favor of the view that the rise of Protestantism in Western Europe promoted economic development. Various arguments imply that Protestants' higher propensity to save is a primary determinant of economic development. However, the existing literature does not take into account that propensity to save is likely to depend on the quality of the financial system. Using a very granular dataset combining religious demographics and financial development at the municipal level, this paper studies the effects of religion on financial development. Employing various instruments, our results show that XXXX. Our data also allow us to distinguish between different kinds of Protestantism: we find that XXX are primarily driving our results. Our estimates imply that the influence of Protestantism on saving are MAGNITUDE.

\textbf{JEL Classifications:} N14, D72, H71

\clearpage

\section{Introduction}

\bibliographystyle{econ}
\bibliography{bibliography}

\end{document}