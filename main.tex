\documentclass[12pt]{extarticle}
\usepackage[utf8]{inputenc}
\usepackage[margin=1in]{geometry}

\usepackage{placeins}   % Fot \FloatBarrier

\usepackage[cmintegrals,cmbraces]{newtxmath}
%\usepackage{ebgaramond}

\usepackage{amsmath}
\usepackage{amsfonts}
%\usepackage{amssymb}

%table numbers within section
\usepackage{chngcntr}
\counterwithin{table}{section}

\usepackage{natbib}
\usepackage{graphicx}
\usepackage{booktabs}
\usepackage{longtable}
\usepackage{caption}

%\frenchspacing
\usepackage{setspace}
\setstretch{1.1}

\usepackage{graphicx}
\usepackage{url}
\usepackage[colorlinks=true, urlcolor = orange, linkcolor=blue, citecolor=purple]{hyperref}

\usepackage{xcolor}
%\usepackage{sectsty}
\usepackage{appendix}

%\subsectionfont{\color{cyan!70!blue}}  % sets colour of chapters
%\sectionfont{\color{blue!60!black}}  % sets colour of sections

%Tables:
\usepackage{dcolumn}
\usepackage{float}
\usepackage{caption}
\usepackage{tabularx}

% Landscape mode
\usepackage{lscape}

% Modelsummary
\usepackage{threeparttable}
\usepackage{booktabs}
\usepackage{siunitx}
%\newcolumntype{d}{S[input-symbols = ()]}
\newcolumntype{d}{S[
    input-open-uncertainty=,
    input-close-uncertainty=,
    parse-numbers = false,
    table-align-text-pre=false,
    table-align-text-post=false
 ]}

%% Regression tables
\usepackage{tabularray}
\usepackage{float}
\usepackage{graphicx}
\usepackage{rotating}
\usepackage[normalem]{ulem}
\UseTblrLibrary{booktabs}
\UseTblrLibrary{siunitx}
\newcommand{\tinytableTabularrayUnderline}[1]{\underline{#1}}
\newcommand{\tinytableTabularrayStrikeout}[1]{\sout{#1}}
\NewTableCommand{\tinytableDefineColor}[3]{\definecolor{#1}{#2}{#3}}

\title{Reunification and Innovation\footnote{We are grateful to XXXX for insightful comments and suggestions.}}
\author{Giacomo Domini \and Bas Machielsen\footnote{Both Utrecht University School of Economics, Utrecht University, Kriekenpitplein 21-22, 3584 EC Utrecht, the Netherlands; e--mail: \href{mailto:a.h.machielsen@uu.nl}{a.h.machielsen@uu.nl}}}
\date{\today}

\begin{document}

\maketitle

\begin{center}
    For the latest version, please \href{http://link.com/paper.pdf}{click here.}
\end{center}

\begin{center} \textbf{Abstract:} \end{center}

\noindent Under Austro-Hungarian rule, the Lombardy and Veneto regions were subject to centralized, autocratic administration. After annexation by the Kingdom of Italy, they became part of a constitutional monarchy, where citizens had greater political participation and representation. We exploit these natural experiments arising from the reunification of Lombardy (1859) and Venetia (1866) with the rest of Italy to study the impact of the accompanying democratization and national reintegration on economic activity through the innovation channel. We combine a spatial regression discontinuity-design with panel data on world industrial expositions to study the impact of reunification on product complexity and innovative activity at the city-level, comparing increases in innovative activity of towns that remained in the Austro-Hungarian empire with towns that were incorporated in Italy. Our results show that [XXX]. 

\textbf{JEL Classifications:} N14, D72, H71

\clearpage

\section{Introduction}

The effect of democratization on economic activity has been debated in many disciplines. Despite some evidence to the contrary, the consensus now seems to be that there likely is a positive influence of democracy on economic activity. Various empirical studies have shown that, in a cross-country context, democratization causally influence economic growth. Theoretically, there are many possible channels through which democratization can lead to economic growth. % Mention some stuff

Empirically, however, there is a lack of evidence on the specific mechanisms at work. In this paper, we propose to investigate one specific mechanism: economic growth through innovative activity. To investigate this channel, we look at the case of Italian reunification, involving the successive integration of the Lombardy (1859) and Veneto (1866) regions into the Kingdom of Italy. Under Austrian rule, Lombardy and Veneto were subject to a centralized, autocratic administration. After annexation by the Kingdom of Italy, they became part of a constitutional monarchy, where its citizens had greater political participation and representation. Local administrations and governance structures were overhauled to align with the new Italian government, which introduced a more centralized and unified system.

Additionally, the Austrian Empire had implemented a policy of Germanization in some areas of Lombardy and Veneto, which affected local language and culture, and with Italian unification, there was a renewed emphasis on the Italian language and culture. The legal system also underwent many changes: after incorporation, the legal and judicial systems in both Lombardy and Veneto shifted from Austrian law to Italian law. This change included the adoption of the Italian legal code and legal practices.

Finally, the Italian government explicitly aimed to modernize and develop the economy of these regions. Investments in infrastructure, transportation, and industry were made to promote economic growth and integration into the wider Italian economy. The adoption of the Italian lira as the official currency replaced the Austrian currency.

We leverage two arguably exogenous border changes involving the incorporation of Lombardy (1859) and Veneto (1866) into Italy to identify the influence of all aforementioned impetuses on economic and innovative activity. Our first treatment, the incorporation of Lombardy, resulted from the termination of the Franco-Austrian War, or the \textit{Second War of Independence} (1859), when the Kingdom of Sardinia-Piedmont, the driving force behind Italian unification, allied with France and went to war against the Austrian Empire. The war ended with the Treaty of Villafranca, in which the French Emperor Napoleon III brokered peace between the warring parties, which led to the annexation of Lombardy, including its capital, Milan, by the Kingdom of Sardinia-Piedmont.

Our second treatment, the incorporation of Veneto, revolves around the termination of the \text{Third War of Independence}, in 1866. This war, waged in parallel to the larger Prussian--Austro-Hungarian war, was terminated by the peace of Prague, which stipulated the Austrian cession of the Veneto region. The Peace of Prague was followed up by the Austrian-Italian Treaty of Vienna, which confirmed the cession of the territory to Italy. However, the peace treaty stated that the annexation of Venetia and Mantua would have become effective only after a referendum, which was held on 21 and 22 October, and the result was an overwhelming success. 

Our identification strategy leverages the randomness involved in the exact determination of the border and the comparability between cities located at similar distances to the new border. 

We also contribute to the democratization literature by providing a data-driven approach to democratization questions on the micro level. Most empirical evidence comes from variation at the country-year level, whereas this study explicitly focuses on the more granular city-level. 


\section{Data \& Methods}

\subsection{Exhibition Data} 
The data employed in this paper are retrieved from the official catalogues of the five Parisian universal exhibitions introduced above (Exposition des produits de l’industrie de toutes les nations, 1855, 1855; Exposition universelle de 1867 à Paris, 1867; Exposition universelle internationale de 1878 à Paris, 1878; Exposition universelle internationale de 1889, à Paris 1889a, Exposition universelle internationale de 1889, à Paris 1889b, Exposition universelle internationale de 1889, à Paris 1889c, Exposition universelle internationale de 1889, à Paris 1889d; Exposition internationale universelle de 1900 à Paris, 1900). This section discusses technical aspects concerning the original data classification, the construction of the database, and the variables computed from exhibition data (RSCA and ECI). More detailed information on these issues is provided in the appendix.

\subsection{Geographical Data}



\subsection{Methods}


\section{Results}

\section{Conclusion}

\bibliographystyle{econ}
\bibliography{bibliography}

\end{document}