\documentclass[12pt]{extarticle}
\usepackage[utf8]{inputenc}
\usepackage[margin=1in]{geometry}

\usepackage{placeins}   % Fot \FloatBarrier

\usepackage[cmintegrals,cmbraces]{newtxmath}
%\usepackage{ebgaramond}

\usepackage{amsmath}
\usepackage{amsfonts}
%\usepackage{amssymb}

%table numbers within section
\usepackage{chngcntr}
\counterwithin{table}{section}

\usepackage{natbib}
\usepackage{graphicx}
\usepackage{booktabs}
\usepackage{longtable}
\usepackage{caption}

%\frenchspacing
\usepackage{setspace}
\setstretch{1.1}

\usepackage{graphicx}
\usepackage{url}
\usepackage[colorlinks=true, urlcolor = orange, linkcolor=blue, citecolor=purple]{hyperref}

\usepackage{xcolor}
%\usepackage{sectsty}
\usepackage{appendix}

%\subsectionfont{\color{cyan!70!blue}}  % sets colour of chapters
%\sectionfont{\color{blue!60!black}}  % sets colour of sections

%Tables:
\usepackage{dcolumn}
\usepackage{float}
\usepackage{caption}
\usepackage{tabularx}

% Landscape mode
\usepackage{lscape}

% Modelsummary
\usepackage{threeparttable}
\usepackage{booktabs}
\usepackage{siunitx}
%\newcolumntype{d}{S[input-symbols = ()]}
\newcolumntype{d}{S[
    input-open-uncertainty=,
    input-close-uncertainty=,
    parse-numbers = false,
    table-align-text-pre=false,
    table-align-text-post=false
 ]}

%% Regression tables
\usepackage{tabularray}
\usepackage{float}
\usepackage{graphicx}
\usepackage{rotating}
\usepackage[normalem]{ulem}
\UseTblrLibrary{booktabs}
\UseTblrLibrary{siunitx}
\newcommand{\tinytableTabularrayUnderline}[1]{\underline{#1}}
\newcommand{\tinytableTabularrayStrikeout}[1]{\sout{#1}}
\NewTableCommand{\tinytableDefineColor}[3]{\definecolor{#1}{#2}{#3}}

\title{Institutions and Innovation: \\ Evidence from the Italian Unification\footnote{We are grateful to the participants to the PPE conference (Utrecht, April 2024), the Applied Economics internal seminar (Utrecht, May 2024), for insightful comments and suggestions.}}
\author{Giacomo Domini \and Bas Machielsen\footnote{Both Utrecht University School of Economics, Utrecht University, Kriekenpitplein 21-22, 3584 EC Utrecht, the Netherlands; e--mail: \href{mailto:a.h.machielsen@uu.nl}{a.h.machielsen@uu.nl}}}
\date{\today}

\begin{document}

\maketitle

\begin{center}
    For the latest version, please \href{http://link.com/paper.pdf}{click here.}
\end{center}

\begin{center} \textbf{Abstract:} \end{center}

%\noindent Under Austro-Hungarian rule, the Lombardy and Veneto regions were subject to centralized, autocratic administration. After annexation by the Kingdom of Italy, they became part of a constitutional monarchy, where citizens had greater political participation and representation. We exploit natural experiments arising from the reunification of Lombardy (1859) and Venetia (1866) with the rest of Italy to study the impact of the accompanying democratization and national reintegration on economic activity through the innovation channel. We combine a spatial regression discontinuity-design with panel data on world industrial expositions to study the impact of reunification on product complexity and innovative activity at the city-level, comparing increases in innovative activity of towns that remained in the Austro-Hungarian empire with towns that were incorporated in Italy. Our results show that [XXX]. 

\noindent Economists have long acknowledged the centrality of innovation in economic growth and of institutions, defined as the “rules of the game”, as a determinant of innovation outcomes. This paper studies the impact on innovation of an important historical episode of institutional change, from the Italian unification process. We leverage quasi-random variation in the establishment of a new border after a war between Piedmont and Austria in 1859; and argue that the Western part of the Lombardo-Venetian kingdom, which was annexed by Piedmont, was exposed to more liberal institutions than the Eastern part, which remained under Austria. We measure innovation using  patent data from Italy and Austria, as well as data from the universal exhibitions taking place in Paris in 1855, 1867, 1878, 1889, and 1900. Preliminary results from regression-discontinuity and difference-in differences regressions show a positive short-run effect of the new institutional setting on innovation.

\textbf{JEL Classifications:} N14, D74, O31 % D72, H71

\textbf{JEL Classifications:} Institutions; innovation; patents; Italian unification

\clearpage

\section*{Extended abstract}
\noindent Economists have long acknowledged the centrality of innovation in economic growth (Schumpeter, 1942; Solow, 1957) and of institutions, defined as the “rules of the game”, as a primary determinant of innovation outcomes (North and Thomas, 1973; North, 1990; Acemoglu et al., 2005). %The importance of institutional frameworks in supporting innovative activities is also emphasized by the literature on national innovation systems, started by Freeman (1987) and further developed by Lundvall (1992) and Nelson (1993).
Nineteenth-century Europe represents a fertile ground for studying the link between institutional change and innovation: on the one hand, the continent was swept by nationalist movements and demands for political and social reform. On the other hand, that century first saw the spread of the First Industrial Revolution to the continent and then the emergence of the Second. This study focuses on an episode from Italy’s unification process in the 1850s and 1860s, namely the annexation of Lombardy to Piedmont from Austria. We argue that this event entailed a substantial shift towards liberal institutions, driven by quasi-random military circumstances. % The historical analysis of this episode is particularly relevant in the current context, where the liberal-democratic global order faces serious challenges.

Our study speaks to different strands of literature. It contributes to the empirical literature on the link between institutions, innovation, and economic growth. Studies on the effect of institutions on economic growth are numerous, starting from Acemoglu et al. (2001) and Rodrik et al. (2004). Fewer have specifically investigated on the effect of institutions on innovation. Donges et al. (2022) provide historical evidence of a positive effect of inclusive institutions on patenting, exploiting variation in the  French occupation of German regions in the Napoleonic era. Some works more narrowly focus on democracy (Aghion et al., 2007; Gao et al., 2017; Acemoglu et al., 2019; Wang et al., 2021), finding an overall positive effect on the generation of new ideas. In addition, our work intends to contribute to the economic-history literature analysing the innovation performance of Italy in the decades following its unification (Barbiellini Amidei et al., 2013; Toninelli and Vasta, 2014; Nuvolari and Vasta 2015, 2017; Nuvolari et al., 2018; Domini, 2023).

The institutional shock we analyse was the outcome of the second war of independence (1859), fought by Piedmont (formally the Kingdom of Sardinia) and France against Austria, with the intent to take from the latter the Lombardo-Venetian Kingdom, broadly corresponding to the present-day regions of Lombardy, Veneto, and Friuli. Austria first entered Piedmont on 27 April 1859, but was pushed back after the arrival of the allied French army. Two months later, the Franco-Piedmontese achieved a decisive victory in Solferino, close to the border between Lombardy and Veneto, after which Napoleon III unexpectedly signed an armistice with Austria. This assigned Lombardy to France, which then transferred it to Piedmont in exchange for Savoy and Nice. This outcome fell short of Piedmont's ambitions, which included also the Eastern part of the kingdom --- so much that prime minister Camillo Benso, count of Cavour, resigned. Veneto only joined what had meanwhile become the Kingdom of Italy (1861) seven years later, in October 1866, after a third war of independence was fought.

We see this as a quasi-random assignment, as the establishment of a new border between Lombardy and Veneto was not anticipated and depended on military and political contingencies. We further argue that Lombardy was ``treated'' with more liberal institutions, as it joined Piedmont, while Veneto remained under repressive Austria. Piedmont was the only Italian state not to repeal the constitution granted in 1848, known as \textit{Statuto Albertino}. This was quite an advanced liberal constitution,  for that time, establishing an elective lower chamber (while the upper one was appointed), and enshrining the principles of equality before the law, individual liberty, freedom of press, and inviolability of private property.
Meanwhile, Austria had a troubled constitutional history: the constitution granted in 1848 was first amended and eventually withdrawn in 1851. A constitution was only re-introduced after the loss of Lombardy, namely the October Diploma (1860), which was soon replaced by the February Patent (1861-1865) and eventually by the December Constitution (1867-1918). Liberal newspaper \textit{Wanderer} (1864) described the February Patent as  ``a constitution without freedom of association, without jury courts, without freedom of press, without equality of confessional rights, lacking a reform of justice and administration.”

While a more advanced constitution may have favoured innovation by establishing stronger individual and property rights, it is important to acknowledge that it was not the only dimension of the ``treatment'' Lombardy was subject to. One that might directly affect Lombardy's innovative performance was a change in the patent system. However, the Piedmontese and Austrian patent systems were similar, as both had been recently reformed (in 1855 and 1852, respectively) along the lines of France's 1844 patent law, establishing a registration system with no novelty examination. Patents were somewhat cheaper on the Italian side, although the difference in fees depended on the duration requested and ranged from zero for a five-year patent to -28.6\% for the maximum duration, 15 years, based on fees reported by Tolhausen (1868). 
Other changes potentially relevant to innovation outcomes were those in market potential (Missiaia, 2016) %-- although this may be expected to be detrimental in the short run, as Lombardy lost access to long-integrated Veneto, as well as to the large Austrian empire, while initially gaining against innovation to  
and in the cultural homogeneity of the new country (Ertug et al., 2021; Mokyr, 2024). %introduces the concept of homophily preference, indicating that similar social preferences can drive collaborative innovation

We measure innovation by means of patent data and international exhibition data. As for the former, we sum Austrian and Piedmontese (Italian, after 1861) patents, to account for the diversion of patenting activity, as Lombardy moved from one state to the other. While patents are a standard measure of innovation, especially in historical settings (Streb, 2023), they suffer from well-known shortcomings, namely they capture invention rather than innovation, and not all inventions are patented, with the propensity to patent widely varying across sectors (Griliches 1990; Nagaoka et al. 2010). As an alternative measure, Moser (2005) introduced the products displayed at international exhibitions. Domini (2019) subsequently argued that these may suffer from opposite drawbacks, compared to patents, as they may represent not necessarily innovative commercialised products. On the bright side, however, exhibits are many more than patents: in 1867, there were 3,841 Italian items on display at the Paris universal exhibition, \textit{vis-à-vis} 164 patents granted to residents in Italy. We use data from the universal exhibitions taking place in Paris in 1855, 1867, 1878, 1889, and 1900. %Notice that the lower frequency of these events implies that this source is rather suited for long-run analysis. 

The empirical strategy makes use of two designs. The first is a cross-sectional regression discontinuity (RD):

\begin{equation*}
    Y_{i} = \alpha + \beta D_i + \gamma Dist_i + \epsilon_i 
\end{equation*}

\noindent where $ D_i $ is a dummy equal to 1 if municipality $ i $ is located in the territory annexed by Piedmont after the war in 1859, $ Dist_i $ is distance from the new border, and the sample includes observations within a certain distance from the border. The second design is a difference-in-differences (DD):

\begin{equation*}
    Y_{it} = \alpha_i + \beta_1 \text{Post}_t + \beta_2 D_i \cdot \text{Post}_t + \epsilon_{it}
\end{equation*}

\noindent where $ Post_t $ indicates years after 1859. The dependent variable $ Y_i $ is, alternatively, the sum of Austrian and Piedmontese (Italian) patents, the count of exhibits, or the top complexity (Hidalgo and Hausmann, 2009) of the products exhibited by municipality $ i $, based on product complexity estimates by Domini (2022), which can be interpreted as a measure of $ i $'s capabilities.

Our preliminary results, both from RD and DD, show a positive short-run effect of Lombardy's annexation on patents, though not on exhibits. It is important to mention that event studies do not reveal pre-trends, which lends credibility to the parallel trends assumption, underpinning the DD design. Planned next steps include a regression with exhibit-based complexity as a dependent variable; a long-run analysis, employing both patent and exhibition data up to the turn of the century; and controlling for co-occurring such as that in market potential.  


\section{Introduction}

% Complementary framing: periphery vs. central part 
The effect of democratization on economic activity has been debated in many disciplines. Despite some evidence to the contrary, the consensus now seems to be that there likely is a positive influence of democracy on economic activity. Various empirical studies have shown that, in a cross-country context, democratization causally influence economic growth. Theoretically, there are many possible channels through which democratization can lead to economic growth. % Mention some stuff

Empirically, however, there is a lack of evidence on the specific mechanisms at work. In this paper, we propose to investigate one specific mechanism: economic growth through innovative activity. To investigate this channel, we look at the case of Italian unification, involving the successive integration of the Lombardy (1859) and Veneto (1866) regions into the Kingdom of Italy. Under Austrian rule, Lombardy and Veneto were subject to a centralized, autocratic administration. After annexation by the Kingdom of Italy, they became part of a constitutional monarchy, where its citizens had greater political participation and representation. Local administrations and governance structures were overhauled to align with the new Italian government, which introduced a more centralized and unified system.

Additionally, the Austrian Empire had implemented a policy of Germanization in some areas of Lombardy and Veneto, which affected local language and culture, and with Italian unification, there was a renewed emphasis on the Italian language and culture. The legal system also underwent many changes: after incorporation, the legal and judicial systems in both Lombardy and Veneto shifted from Austrian law to Italian law. This change included the adoption of the Italian legal code and legal practices.
% Make these legal practices and patent law specifically more precise

Finally, the Italian government explicitly aimed to modernize and develop the economy of these regions. Investments in infrastructure, transportation, and industry were made to promote economic growth and integration into the wider Italian economy. The adoption of the Italian lira as the official currency replaced the Austrian currency.

We leverage two arguably exogenous border changes involving the incorporation of Lombardy (1859) and Veneto (1866) into Italy to identify the influence of all aforementioned impetuses on economic and innovative activity. Our first treatment, the incorporation of Lombardy, resulted from the termination of the Franco-Austrian War, or the \textit{Second War of Independence} (1859), when the Kingdom of Sardinia-Piedmont, the driving force behind Italian unification, allied with France and went to war against the Austrian Empire. The war ended with the Treaty of Villafranca, in which the French Emperor Napoleon III brokered peace between the warring parties, which led to the annexation of Lombardy, including its capital, Milan, by the Kingdom of Sardinia-Piedmont.

Our second treatment, the incorporation of Veneto, revolves around the termination of the \text{Third War of Independence}, in 1866. This war, waged in parallel to the larger Prussian--Austro-Hungarian war, was terminated by the peace of Prague, which stipulated the Austrian cession of the Veneto region. The Peace of Prague was followed up by the Austrian-Italian Treaty of Vienna, which confirmed the cession of the territory to Italy. However, the peace treaty stated that the annexation of Venetia and Mantua would have become effective only after a referendum, which was held on 21 and 22 October, and the result was an overwhelming success. 

% How did the war go and where was it fought? No confounding factor?

Our identification strategy leverages the randomness involved in the exact determination of the border and the comparability between cities located at similar distances to the new border. 

We also contribute to the democratization literature by providing a data-driven approach to democratization questions on the micro level. Most empirical evidence comes from variation at the country-year level, whereas this study explicitly focuses on the more granular city-level. 

\section{Historical background}
We focus on one important historical episode, namely Italy’s unification process in the 1850s and 1860s. After being the richest part of the Western world until the Middle Ages, the Italian peninsula underwent a steady decline throughout the modern era. Lacking political unity since the fall of the Western Roman Empire and missing direct access to ocean routes, the country suffered in an age characterised by the development of strong nation-states and by the redirection of trade routes from the Mediterranean to the Atlantic. From the end of the 15th century onwards, the country fell prey of foreign powers such as France, Spain, and Austria, which occupied large parts of it. After the French revolution and the Napoleonic era, a sentiment in favour of the independence and unification of the country grew and became prominent among the country's intellectual elites. This found political and military guidance in the ambitious and (relatively) liberal North-Western Italian state of Piedmont (formally the Kingdom of Sardinia, although its political and economic centre was not on the island but on the continent). In 1848-1849, after uprisings against Austrian rule in the cities of Milan and Venice, this state led a coalition of volunteers from the entire peninsula in an unsuccessful war in the Austrian-occupied Lombardo-Venetian Kingdom, going under the name of First War of Independence in Italian historiography. 

In 1859, after securing an alliance with France, Piedmont fought a new war against Austria, aimed at annexing the Lombardo-Venetian Kingdom, known as the Second War of Independence. This war was a partial success: while the Western provinces (Lombardy proper) were easily conquered, the French-Piedmontese expedition found resistance while trying to advance into the Eastern part of the Austrian domains (the Venetian provinces). This led the French emperor Napoleon III to seek an armistice with Austria. Lombardy was thus annexed by the Kingdom of Sardinia, together with some other Italian states (the Grand-Duchy of Tuscany, the Duchies of Parma and Modena, and the Northern part of Papal States), while Venetia remained under Austrian rule. In 1861, after the annexation of the Southern-Italian Kingdom of the Two Sicilies, the King of Sardinia, Vittorio Emanuele II, proclaimed himself King of Italy. Venetia was eventually taken from Austria in 1866 during the Third War of Independence, fought as Austria was engaged in the Austro-Prussian war. After the conquest of Rome in 1870, most of present-day Italy was unified under the new Kingdom of Italy, with the exception of some border regions in the North-East, which were eventually annexed after Italy's victory against Austria in the First World War.

The institutional shock we analyse took place within the context of Italy's fights for national independence and unification, the \textit{Risorgimento}. It was the outcome of the second war of independence (1859), waged by Piedmont (formally the Kingdom of Sardinia) and France against Austria, with the intent to take from the latter the Lombardo-Venetian Kingdom, broadly corresponding to the present-day regions of Lombardy, Veneto, and Friuli. Ten years after the unsuccessful first war of independence (1848-1849), in which an Italian coalition led by Piedmont had already tried to free the Lombardo-Venetian Kingdom, % but was defeated by Austrian commander Radetzky, 
Piedmont planned a new attempt, securing the military alliance of Napoleon III's France. The agreement stipulated that France would help Piedmont in case of Austrian aggression and, in case of victory, Piedmont would annex the Lombardo-Venetian Kingdom, while France would receive Savoy and Nice, then parts of the Kingdom of Sardinia. After a diplomatic escalation, war broke out: Austria entered Piedmont on 27 April 1859, but was pushed back after the arrival of the French army. Two months later, the Franco-Piedmontese achieved a decisive victory in Solferino, close to the border between Lombardy and Veneto, after which Napoleon III unexpectedly signed an armistice with Austria. This assigned Lombardy to France, which then transferred it to Piedmont in exchange for Savoy and Nice. This outcome fell short of Piedmont's ambitions, so much that prime minister Camillo Benso, count of Cavour, resigned. Veneto only joined what had meanwhile become the Kingdom of Italy (1861) seven years later, in October 1866, after a third war of independence was fought.

We see this as a quasi-random assignment, as the establishment of a new border between Lombardy and Veneto was not anticipated and depended on military and political developments. We further argue that Lombardy was ``treated'' with more liberal institutions, as it joined Piedmont, while Veneto remained under repressive Austria. Piedmont was the most politically advanced state of pre-unification Italy. A stable monarchy of the House of Savoy, it was the only Italian state not to repeal the constitution granted in 1848, known as \textit{Statuto Albertino}. It also advocated free trade, and established compulsory universal schooling in 1859. The \textit{Statuto} was a liberal constitution, quite advanced for that time, establishing an elective lower chamber (while the upper one was appointed), and enshrining the principles of equality before the law, individual liberty, freedom of press, and inviolability of private property.

Meanwhile, Austria had a troubled constitutional history: the %Pillersdorf 
constitution granted in 1848 was first amended and eventually withdrawn in 1851. A constitution was only re-introduced after the loss of Lombardy, namely the October Diploma (1860), which was soon replaced by the February Patent (1861-1865) and eventually by the December Constitution (1867-1918). The February Patent established a bicameral system, with one chamber appointed and the other indirectly elected via the provincial diets; %(i.e. devolved parliaments); 
but, in the words of the liberal newspaper \textit{Wanderer} (1864), it was ``a constitution without freedom of association, without jury courts, without freedom of press, without equality of confessional rights, lacking a reform of justice and administration.”

While a more advanced constitution may have favoured innovation by establishing stronger individual and property rights, it is important to acknowledge that it was not the only dimension of the ``treatment'' Lombardy was subject to. One that might directly affect Lombardy's innovative performance was a change in the patent system. However, the Piedmontese and Austrian patent systems were quite similar, as both had been recently reformed (in 1855 and 1852, respectively) along the lines of France's 1844 patent law, establishing a registration system with no novelty examination. Patents were somewhat cheaper on the Italian side, although the difference in fees depended on the duration requested and ranged from zero for a five-year patent to -28.6\% for the maximum duration, 15 years, based on fees reported by Tolhausen (1868). 
Other changes potentially relevant to innovation outcomes were those in market potential (Missiaia, 2016) %-- although this may be expected to be detrimental in the short run, as Lombardy lost access to long-integrated Veneto, as well as to the large Austrian empire, while initially gaining against innovation to  
and in the cultural homogeneity of the new country (Ertug et al., 2021; Mokyr, 2024). %introduces the concept of homophily preference, indicating that similar social preferences can drive collaborative innovation


\section{Data \& Methods}

\subsection{Exhibition Data} 
The data employed in this paper are retrieved from the official catalogues of the five Parisian universal exhibitions introduced above (Exposition des produits de l’industrie de toutes les nations, 1855, 1855; Exposition universelle de 1867 à Paris, 1867; Exposition universelle internationale de 1878 à Paris, 1878; Exposition universelle internationale de 1889, à Paris 1889a, Exposition universelle internationale de 1889, à Paris 1889b, Exposition universelle internationale de 1889, à Paris 1889c, Exposition universelle internationale de 1889, à Paris 1889d; Exposition internationale universelle de 1900 à Paris, 1900). This section discusses technical aspects concerning the original data classification, the construction of the database, and the variables computed from exhibition data (RSCA and ECI). More detailed information on these issues is provided in the appendix.

\subsection{Geographical Data}

We start off with a shapefile of Italian municipalities. Then, we superimpose a hand-made digitization of the border between Veneto and Lombardo in 1859. The shapefiles are available on [repository]. We then name match the exhibition data to the names of the municipality using a firstly a rule-based approach based on exact, then fuzzy string matching with a relatively strict criterion. If still no match has been found, we use a LLM (OpenAI 2023) to find a match. The specific query we use is described in [repository]. 

\subsection{Data Pipeline}

% also incorporate with geographical data, we can include 0 observations to the dataset

% mention the packages I use in here 

Our pipeline starts from the digitized exposition data from an exposition at time $t$. Using a layout detection deep learning algorithm, we identify (i) the particular class of products being exhibited, and for each class, we then identify the observations represented by row entries. 

Observations are defined as meaning an exhibitor $g$ (identified by a name) from town $i$, exhibiting a product from class $j$ at time of the exhibition $t$. Initially, we end up with raw strings containing this information. We use a rule-based approach to extract the town $i$ from the string, and then match it using a pre-trained language model-based approach to a predetermined list of city names, allowing us to geo-code our observations according to the city of origin. 

Our baseline analyses involve aggregating this data to the city-class-year level $ijt$: we thus end up with a count variable for number of exhibitors in each class $j$ in exhibition year $t$. We augment this dataset by all towns in the affected areas that had no exhibitors in clas $j$ in exhibition year $t$. 



\subsection{Methods}

% think about: construct a complexity measure from the verbal descriptions of the product

% complexity is a exhibitor-region-class level variable, but innovative activity or something is a region-class level variable

Our point of reference encompasses a cross-sectional design around municipalities $i$:

\begin{equation*}
    Y_{i, 1867} = \alpha + \beta D_i + \epsilon_i 
\end{equation*}

where $D_i$ is a treatment indicator, equal to 1 if city $i$ has become part of Italy in 1859 and 1866 respectively. We might also consider a local regression discontinuity design, where $D_i$ is defined as \textit{distance to the border}, multiplied by -1 if the city is located in the Austro-Hungarian Empire in 1867. \footnote{We can compare for before-treatment similarity by running $Y_{i, 1855} = \alpha_0 + \alpha_1 D_i + u_i$}

A second approach might be leveraging the panel data from different expositions at times $t \in \{ 1855, 1867\}$. In that case, the model we can estimate for $Y$ in city $i$ at time $t$: Methods

\begin{equation*}
    Y_{it} = \alpha_i + \beta_1 \text{Post}_t + \beta_2 D_i \cdot \text{Post}_t + \epsilon_{it}\footnote{We could also replace the $\alpha_i$ by a "mean difference" $D_i$, making it a standard difference-in-difference. The model in the equation, is however more general.}
\end{equation*}

in which we control for level differences between cities $i$ by the $\alpha_i$'s, and we compare villages at the same distance from the border before and after the unification. Our coefficient of interest is $\beta_2$, the difference between the treatment and control groups after the treatment. $D_i$ 

In this regression, we could add distance to the border as a control variable, and potentially we can also weight the observations by the inverse distance to the border. Then, it comes closer to a geographical RDD design. 

Finally, we can also opt for an explicit geographical local regression discontinuity estimate combine with a difference in difference methodology: 

\begin{equation*}
    Y_{it} - Y_{it-1} = \alpha + \beta D_i + \epsilon_{it}
\end{equation*}

where in the geographical RDD design, we define $D_i$ as being the distance to the border. 


% Take into account, super-city level of aggregation if we want
% Also; city 

% Patent data in the future

\section{Results}


\subsection{RD Estimates of Unification on Innovative Activity}

In Table \ref{tab:rd_analysis_number}, the estimates of unification on innovations are reported. In this analysis, we look at the cross-sectional innovative activity in municipality $i$ and we find there is a discontinuous increase in innovative activity in municipalities that are part of the newly-annexed Veneto region compared to the earlier-annexed Lombardy region. 

\begin{center}
    [Table \ref{tab:rd_analysis_number} here]
\end{center}

\subsection{DiD Estimates of Unification on Innovative Activity}

In Table \ref{tab:did_analysis_number}, we analyze the influence of unification on innovative activity using a difference in difference strategy, comparing the innovations in 1855 with the innovations in 1867 in both region. The coefficient of interest is Year (1867) x Veneto, the estimate of the effect of reunification relative to the counterfactual increase in innovation in Lombardy. This analysis shows that 1867 had a higher frequency of innovations overall, but there was no discernable difference between Veneto and Lombardy. Specifically, in this analysis, we find no discernable effect of reunification on the count of innovations. In the OLS models, the treatment effect is close to zero. In the Poisson models, the point estimate is slightly negative but this is likely to due to sampling error. 

\begin{center}
    [Table \ref{tab:did_analysis_number} here]
\end{center}

\section{Mechanisms}

% Democratization
%% Almost equally undemocratic - Austro-Hungary vs. Italy
%% What kind of variation to use? 

% Homophily 
%% Ethnic and linguistic coherence (Oded Galor and descendants)
%% Changes in the market potential of different regions
%% Tariffs and Trade barriers? 
%% After becoming part of Italy, entire A-H Market became inaccessible
%%% Railway and Road Data in conjunction 
%%% For some regions this may be bad
%%% But for others not that bad because there were already huge transport costs

% In 1867
% Veneto joined in 1866
% Lombardo joined in 1859 

% Two exercises now: complexity measure on the left hand side
% Have a look at the austrian patent data and Scrape


% New file for Italy 1878, 1889 and 1900

\section{Conclusion}

\bibliographystyle{econ}
\bibliography{bibliography}

\clearpage

\appendix

\section{Tables Main Text}

\begin{table}[!h]

\caption{\label{tab:rd_analysis_number}Estimates of Italian Unification on Innovative Activity}
\centering
\resizebox{\linewidth}{!}{
\begin{threeparttable}
\begin{tabular}[t]{lllll}
\toprule
\multicolumn{1}{c}{ } & \multicolumn{2}{c}{Log Innovations} & \multicolumn{2}{c}{Ihs Innovations} \\
\cmidrule(l{3pt}r{3pt}){2-3} \cmidrule(l{3pt}r{3pt}){4-5}
\multicolumn{1}{c}{ } & \multicolumn{1}{c}{No FE} & \multicolumn{1}{c}{FE} & \multicolumn{1}{c}{No FE} & \multicolumn{1}{c}{FE} \\
\cmidrule(l{3pt}r{3pt}){2-2} \cmidrule(l{3pt}r{3pt}){3-3} \cmidrule(l{3pt}r{3pt}){4-4} \cmidrule(l{3pt}r{3pt}){5-5}
  & (1) & (2) & (3) & (4)\\
\midrule
Estimate & 0.140** & 0.149** & 0.175** & 0.189**\\
SE (BC) & (0.063) & (0.052) & (0.079) & (0.065)\\
Mean DV Treated 50km & 0.083 & 0.083 & 0.103 & 0.103\\
Mean DV Control 50km & 0.062 & 0.062 & 0.078 & 0.078\\
N (Treated) & 667 & 667 & 667 & 667\\
N (Control) & 1549 & 1549 & 1549 & 1549\\
Bandwidth & 41406.168 & 45338.659 & 42289.180 & 46458.038\\
\bottomrule
\end{tabular}
\begin{tablenotes}[para]
\item \textit{Note: } 
\item Table showing coefficient estimates and bias-corrected standard errors of the impact of Italian Unification on innovative activity. The dependent variable is log or ihs no. of innovations and the independent (running) variable is distance to the border. The bandwidth is estimated using the MSE-optimal bandwidth from \cite{cattaneo2019practical}. The estimates control for area, angle to border and innovation. Models (2) and (4) are conditional on province fixed effects. *: p < 0.10, **: p < 0.05, ***: p < 0.01.
\end{tablenotes}
\end{threeparttable}}
\end{table}

\clearpage

\begin{table}[!h]

\caption{\label{tab:did_analysis_number}Difference-in-difference Estimates of Unification on Innovation}
\centering
\fontsize{9}{11}\selectfont
\begin{threeparttable}
\begin{tabular}[t]{lcccc}
\toprule
\multicolumn{1}{c}{ } & \multicolumn{2}{c}{OLS} & \multicolumn{2}{c}{Poisson} \\
\cmidrule(l{3pt}r{3pt}){2-3} \cmidrule(l{3pt}r{3pt}){4-5}
  & (1) & (2) & (3) & (4)\\
\midrule
Year (1867) & \num{0.037}*** & \num{0.037}*** & \num{1.345}*** & \num{1.345}***\\
 & (\num{0.006}) & (\num{0.006}) & (\num{0.225}) & (\num{0.226})\\
Veneto & \num{-0.029} & \num{0.063} & \num{-1.248} & \num{1.111}\\
 & (\num{0.021}) & (\num{0.039}) & (\num{0.873}) & (\num{0.741})\\
Year (1867) x Veneto & \num{0.007} & \num{0.007} & \num{-0.130} & \num{-0.130}\\
 & (\num{0.013}) & (\num{0.013}) & (\num{0.367}) & (\num{0.368})\\
Elevation & \num{0.000}*** & \num{0.000}*** & \num{-0.003}*** & \num{-0.004}***\\
 & (\num{0.000}) & (\num{0.000}) & (\num{0.001}) & (\num{0.001})\\
Longitude & \num{-0.021}* & \num{-0.049}** & \num{0.111} & \num{-0.735}\\
 & (\num{0.011}) & (\num{0.022}) & (\num{0.219}) & (\num{0.622})\\
Latitude & \num{0.081}*** & \num{0.083}** & \num{2.126}*** & \num{2.756}***\\
 & (\num{0.026}) & (\num{0.036}) & (\num{0.713}) & (\num{0.979})\\
Area & \num{0.004}*** & \num{0.004}*** & \num{0.021}*** & \num{0.040}***\\
 & (\num{0.001}) & (\num{0.001}) & (\num{0.003}) & (\num{0.005})\\
Angle to Border & \num{0.000} & \num{0.000} & \num{-0.005}* & \num{-0.003}\\
 & (\num{0.000}) & (\num{0.000}) & (\num{0.002}) & (\num{0.002})\\
\midrule
N & \num{4432} & \num{4432} & \num{4432} & \num{4432}\\
Province FE & No & Yes & No & Yes\\
\bottomrule
\multicolumn{5}{l}{\rule{0pt}{1em}* p $<$ 0.1, ** p $<$ 0.05, *** p $<$ 0.01}\\
\end{tabular}
\begin{tablenotes}[para]
\item \textit{Note: } 
\item Dependent variables: Number of innovations in municipality $i$. The coefficient of interest is the Year x Group{Veneto} variable. The control variables are latitude, longitude, elevation, and the analysis is conditional on provinde fixed-effects. Standard errors are clustered at the municipality level.
\end{tablenotes}
\end{threeparttable}
\end{table}

\clearpage

\begin{table}[!h]

\caption{\label{tab:rd_analysis_placebo}Placebo Test of Italian Unification on Innovative Activity}
\centering
\resizebox{\linewidth}{!}{
\begin{threeparttable}
\begin{tabular}[t]{lllll}
\toprule
\multicolumn{1}{c}{ } & \multicolumn{2}{c}{Log Innovations} & \multicolumn{2}{c}{Ihs Innovations} \\
\cmidrule(l{3pt}r{3pt}){2-3} \cmidrule(l{3pt}r{3pt}){4-5}
\multicolumn{1}{c}{ } & \multicolumn{1}{c}{No FE} & \multicolumn{1}{c}{FE} & \multicolumn{1}{c}{No FE} & \multicolumn{1}{c}{FE} \\
\cmidrule(l{3pt}r{3pt}){2-2} \cmidrule(l{3pt}r{3pt}){3-3} \cmidrule(l{3pt}r{3pt}){4-4} \cmidrule(l{3pt}r{3pt}){5-5}
  & (1) & (2) & (3) & (4)\\
\midrule
Estimate & -0.025* & -0.020 & -0.035* & -0.023\\
SE (BC) & (0.012) & (0.013) & (0.017) & (0.015)\\
Mean DV Treated 50km & 0.015 & 0.015 & 0.018 & 0.018\\
Mean DV Control 50km & 0.007 & 0.007 & 0.009 & 0.009\\
N (Treated) & 667 & 667 & 667 & 667\\
N (Control) & 1549 & 1549 & 1549 & 1549\\
Bandwidth & 23619.409 & 18088.229 & 24106.902 & 18507.961\\
\bottomrule
\end{tabular}
\begin{tablenotes}[para]
\item \textit{Note: } 
\item Table showing coefficient estimates and bias-corrected standard errors of a placebo test, studying the impact of Italian Unification on innovative activity before unification took place. The dependent variable is log or ihs no. of innovations and the independent (running) variable is distance to the border. The bandwidth is estimated using the MSE-optimal bandwidth from \cite{cattaneo2019practical}. The estimates control for area, angle to border and innovation. Models (2) and (4) are conditional on province fixed effects. *: p < 0.10, **: p < 0.05, ***: p < 0.01.
\end{tablenotes}
\end{threeparttable}}
\end{table}

\clearpage


\section*{Tables Appendix}

\begin{table}[!h]

\caption{\label{tab:rd_analysis_rank}Estimates of Italian Unification on Innovative Activity}
\centering
\resizebox{\linewidth}{!}{
\begin{threeparttable}
\begin{tabular}[t]{lll}
\toprule
\multicolumn{1}{c}{ } & \multicolumn{1}{c}{No FE} & \multicolumn{1}{c}{FE} \\
\cmidrule(l{3pt}r{3pt}){2-2} \cmidrule(l{3pt}r{3pt}){3-3}
  & (1) & (2)\\
\midrule
Estimate & 496.694*** & 510.245***\\
SE (BC) & (144.114) & (112.830)\\
Mean DV Treated 50km & 227.098 & 227.098\\
Mean DV Control 50km & 254.967 & 254.967\\
N (Treated) & 667 & 667\\
N (Control) & 1549 & 1549\\
Bandwidth & 33300.049 & 36859.734\\
\bottomrule
\end{tabular}
\begin{tablenotes}[para]
\item \textit{Note: } 
\item Table showing coefficient estimates and bias-corrected standard errors of the impact of Italian Unification on innovative activity. The dependent variable is Ranking(no. of innovations) and the independent (running) variable is distance to the border. The bandwidth is estimated using the MSE-optimal bandwidth from \cite{cattaneo2019practical}. The estimates control for area, angle to border and innovation. Models (2) and (4) are conditional on province fixed effects. *: p < 0.10, **: p < 0.05, ***: p < 0.01.
\end{tablenotes}
\end{threeparttable}}
\end{table}

\clearpage

\begin{table}[!h]

\caption{\label{tab:did_analysis_rank}Difference-in-difference Estimates of Unification on Innovation (Ranking)}
\centering
\fontsize{9}{11}\selectfont
\begin{threeparttable}
\begin{tabular}[t]{lcccc}
\toprule
\multicolumn{1}{c}{ } & \multicolumn{2}{c}{OLS} & \multicolumn{2}{c}{Poisson} \\
\cmidrule(l{3pt}r{3pt}){2-3} \cmidrule(l{3pt}r{3pt}){4-5}
  & (1) & (2) & (3) & (4)\\
\midrule
Year (1867) & \num{131.938}*** & \num{131.938}*** & \num{1.352}*** & \num{1.352}***\\
 & (\num{21.298}) & (\num{21.342}) & (\num{0.237}) & (\num{0.237})\\
Veneto & \num{-111.120}** & \num{176.963}** & \num{-1.302}** & \num{1.049}*\\
 & (\num{49.439}) & (\num{85.169}) & (\num{0.590}) & (\num{0.589})\\
Year (1867) x Veneto & \num{-39.857} & \num{-39.857} & \num{-0.349} & \num{-0.349}\\
 & (\num{34.806}) & (\num{34.877}) & (\num{0.393}) & (\num{0.394})\\
Elevation & \num{-0.283}*** & \num{-0.332}*** & \num{-0.002}*** & \num{-0.003}***\\
 & (\num{0.052}) & (\num{0.058}) & (\num{0.000}) & (\num{0.001})\\
Longitude & \num{-5.918} & \num{-33.508} & \num{0.251} & \num{0.154}\\
 & (\num{21.974}) & (\num{49.318}) & (\num{0.164}) & (\num{0.469})\\
Latitude & \num{171.749}*** & \num{245.251}*** & \num{1.563}*** & \num{2.552}***\\
 & (\num{51.876}) & (\num{85.851}) & (\num{0.518}) & (\num{0.824})\\
Area & \num{6.111}*** & \num{6.599}*** & \num{0.017}*** & \num{0.027}***\\
 & (\num{1.192}) & (\num{1.254}) & (\num{0.002}) & (\num{0.004})\\
Angle to Border & \num{-0.200} & \num{-0.225} & \num{-0.004}** & \num{-0.003}*\\
 & (\num{0.127}) & (\num{0.141}) & (\num{0.002}) & (\num{0.002})\\
\midrule
N & \num{4432} & \num{4432} & \num{4432} & \num{4432}\\
Province FE & No & Yes & No & Yes\\
\bottomrule
\multicolumn{5}{l}{\rule{0pt}{1em}* p $<$ 0.1, ** p $<$ 0.05, *** p $<$ 0.01}\\
\end{tabular}
\begin{tablenotes}[para]
\item \textit{Note: } 
\item Dependent variables: Rank(number of innovations) in municipality $i$, where higher is more innovative. The coefficient of interest is the Year x Group{Veneto} variable. The control variables are latitude, longitude, elevation, and the analysis is conditional on provinde fixed-effects. Standard errors are clustered at the municipality level.
\end{tablenotes}
\end{threeparttable}
\end{table}

\clearpage

\begin{table}[!h]

\caption{\label{tab:rd_analysis_dummy}Estimates of Italian Unification on Innovative Activity}
\centering
\resizebox{\linewidth}{!}{
\begin{threeparttable}
\begin{tabular}[t]{lll}
\toprule
\multicolumn{1}{c}{ } & \multicolumn{1}{c}{No FE} & \multicolumn{1}{c}{FE} \\
\cmidrule(l{3pt}r{3pt}){2-2} \cmidrule(l{3pt}r{3pt}){3-3}
  & (1) & (2)\\
\midrule
Estimate & 0.114*** & 0.117***\\
SE (BC) & (0.033) & (0.026)\\
Mean DV Treated 50km & 0.052 & 0.052\\
Mean DV Control 50km & 0.058 & 0.058\\
N (Treated) & 667 & 667\\
N (Control) & 1549 & 1549\\
Bandwidth & 33151.481 & 36876.632\\
\bottomrule
\end{tabular}
\begin{tablenotes}[para]
\item \textit{Note: } 
\item Table showing coefficient estimates and bias-corrected standard errors of the impact of Italian Unification on innovative activity. The dependent variable is Dummy(no. of innovations > 0) and the independent (running) variable is distance to the border. The bandwidth is estimated using the MSE-optimal bandwidth from \cite{cattaneo2019practical}. The estimates control for area, angle to border and innovation. Models (2) and (4) are conditional on province fixed effects. *: p < 0.10, **: p < 0.05, ***: p < 0.01.
\end{tablenotes}
\end{threeparttable}}
\end{table}

\clearpage

\begin{table}[!h]

\caption{\label{tab:did_analysis_dummy}Difference-in-difference Estimates of Unification on Indicator Innovation}
\centering
\fontsize{9}{11}\selectfont
\begin{threeparttable}
\begin{tabular}[t]{lcccc}
\toprule
\multicolumn{1}{c}{ } & \multicolumn{2}{c}{OLS} & \multicolumn{2}{c}{Poisson} \\
\cmidrule(l{3pt}r{3pt}){2-3} \cmidrule(l{3pt}r{3pt}){4-5}
  & (1) & (2) & (3) & (4)\\
\midrule
Year (1867) & \num{0.030}*** & \num{0.030}*** & \num{1.371}*** & \num{1.371}***\\
 & (\num{0.005}) & (\num{0.005}) & (\num{0.242}) & (\num{0.243})\\
Veneto & \num{-0.026}** & \num{0.040}** & \num{-1.320}** & \num{1.053}*\\
 & (\num{0.011}) & (\num{0.019}) & (\num{0.599}) & (\num{0.595})\\
Year (1867) x Veneto & \num{-0.009} & \num{-0.009} & \num{-0.359} & \num{-0.359}\\
 & (\num{0.008}) & (\num{0.008}) & (\num{0.401}) & (\num{0.402})\\
Elevation & \num{0.000}*** & \num{0.000}*** & \num{-0.002}*** & \num{-0.003}***\\
 & (\num{0.000}) & (\num{0.000}) & (\num{0.000}) & (\num{0.001})\\
Longitude & \num{-0.001} & \num{-0.007} & \num{0.257} & \num{0.168}\\
 & (\num{0.005}) & (\num{0.011}) & (\num{0.165}) & (\num{0.473})\\
Latitude & \num{0.039}*** & \num{0.056}*** & \num{1.577}*** & \num{2.582}***\\
 & (\num{0.012}) & (\num{0.020}) & (\num{0.524}) & (\num{0.834})\\
Area & \num{0.001}*** & \num{0.001}*** & \num{0.017}*** & \num{0.027}***\\
 & (\num{0.000}) & (\num{0.000}) & (\num{0.002}) & (\num{0.004})\\
Angle to Border & \num{0.000} & \num{0.000} & \num{-0.004}** & \num{-0.003}*\\
 & (\num{0.000}) & (\num{0.000}) & (\num{0.002}) & (\num{0.002})\\
\midrule
N & \num{4432} & \num{4432} & \num{4432} & \num{4432}\\
Province FE & No & Yes & No & Yes\\
\bottomrule
\multicolumn{5}{l}{\rule{0pt}{1em}* p $<$ 0.1, ** p $<$ 0.05, *** p $<$ 0.01}\\
\end{tabular}
\begin{tablenotes}[para]
\item \textit{Note: } 
\item Dependent variables: Dummy(Number of innovations > 1) in municipality $i$. The coefficient of interest is the Year x Group{Veneto} variable. The control variables are latitude, longitude, elevation, and the analysis is conditional on provinde fixed-effects. Standard errors are clustered at the municipality level.
\end{tablenotes}
\end{threeparttable}
\end{table}

\clearpage

\end{document}