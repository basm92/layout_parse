\usetheme{default}
\setbeamertemplate{navigation symbols}{}
\setbeamertemplate{footline}
{
  \leavevmode%
  \hbox{%
  \begin{beamercolorbox}[wd=.40\paperwidth,ht=2.5ex,dp=1ex,center]{author in head/foot}%
    \usebeamerfont{author in head/foot}\insertshortauthor
  \end{beamercolorbox}%
  \begin{beamercolorbox}[wd=.50\paperwidth,ht=2.5ex,dp=1ex,center]{title in head/foot}%
    \usebeamerfont{title in head/foot}\insertshorttitle
  \end{beamercolorbox}%
  \begin{beamercolorbox}[wd=.10\paperwidth,ht=2.5ex,dp=1ex,center]{date in head/foot}%
    \insertframenumber{} /\inserttotalframenumber\hspace*{1ex}
  \end{beamercolorbox}}%
  \vskip0pt%
}
\makeatletter

\usepackage[T1]{fontenc}
%\usepackage[utf8]{inputenc}
%\usepacage[italian]{babel}
\usepackage{amsmath} % allows to use \text in arrays
\usepackage{afterpage} % Execute command after the next page break
\usepackage{syntonly}
%\syntaxonly
\usepackage{changepage}
\usepackage{comment}

%Tables and figures
\usepackage{tabularx} % introduces column type X
\usepackage{array}
\usepackage{rotating} % per ruotare tabelle grandi
\usepackage{longtable}
\usepackage{siunitx} % gestisce in modo molto potente e flessibile anche la resa tipografica dei numeri nelle tabelle, definendo un nuovo tipo di colonna S specifica per dati numerici
\sisetup{output-decimal-marker={.},input-symbols = ()} % "(" and ")" are ordinary inputs
\setbeamertemplate{caption}[numbered]
\setbeamerfont{framesubtitle}{size=\large}
\usepackage[compatibility=false]{caption}
\captionsetup{tableposition=top,figureposition=bottom,font=scriptsize}
\usepackage{subcaption}
\captionsetup[sub]{font=scriptsize,labelfont={bf,sf}}
\usepackage{booktabs} % per tabelle
\usepackage{graphicx} % per figure
\usepackage{pgfplots}
\pgfplotsset{width=\textwidth,compat=1.11}
\usepackage{adjustbox}
\usepackage{tikz}
\usetikzlibrary{calc}
\graphicspath{{../paper/}{../figures/}{../}}
\usepackage{threeparttable}

%New command for captions below the table (in addition to those above)
\newcommand\fnote[1]{\captionsetup{format=plain,font=footnotesize,justification=justified}\caption*{#1}}
\newcommand{\B}[1]{{\color{blue} #1}}
\newcommand{\G}[1]{{\color{gray} #1}}
\hypersetup{colorlinks=true, breaklinks=false, linkcolor=gray, menucolor=gray, urlcolor=blue, citecolor=gray}

%Symbols
\usepackage{pifont}% http://ctan.org/pkg/pifont
\newcommand{\cmark}{\ding{51}}%
\newcommand{\xmark}{\ding{55}}%

%% Referencing
\usepackage{hyperref}
%\usepackage{breqn}
%\usepackage{amsmath}

%% Citations
%\usepackage[authoryear]{natbib}
%\usepackage{hyperref}
%\citestyle{apacite}
%\def\bibfont{\small}

%Biblio
\usepackage[style=authoryear-comp,backend=bibtex,uniquename=init,firstinits,maxcitenames=2,natbib=true,url=false,doi=false,eprint=false]{biblatex}
\bibliography{bibliography}