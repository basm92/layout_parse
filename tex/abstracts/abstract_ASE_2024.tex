\section*{Extended abstract}
\noindent Economists have long acknowledged the centrality of innovation in economic growth (Schumpeter, 1942; Solow, 1957) and of institutions, defined as the “rules of the game”, as a primary determinant of innovation outcomes (North and Thomas, 1973; North, 1990; Acemoglu et al., 2005). %The importance of institutional frameworks in supporting innovative activities is also emphasized by the literature on national innovation systems, started by Freeman (1987) and further developed by Lundvall (1992) and Nelson (1993).
Nineteenth-century Europe represents a fertile ground for studying the link between institutional change and innovation: on the one hand, the continent was swept by nationalist movements and demands for political and social reform. On the other hand, that century first saw the spread of the First Industrial Revolution to the continent and then the emergence of the Second. This study focuses on an episode from Italy’s unification process in the 1850s and 1860s, namely the annexation of Lombardy to Piedmont from Austria. We argue that this event entailed a substantial shift towards liberal institutions, driven by quasi-random military circumstances. % The historical analysis of this episode is particularly relevant in the current context, where the liberal-democratic global order faces serious challenges.

Our study speaks to different strands of literature. It contributes to the empirical literature on the link between institutions, innovation, and economic growth. Studies on the effect of institutions on economic growth are numerous, starting from Acemoglu et al. (2001) and Rodrik et al. (2004). Fewer have specifically investigated on the effect of institutions on innovation. Donges et al. (2022) provide historical evidence of a positive effect of inclusive institutions on patenting, exploiting variation in the  French occupation of German regions in the Napoleonic era. Some works more narrowly focus on democracy (Aghion et al., 2007; Gao et al., 2017; Acemoglu et al., 2019; Wang et al., 2021), finding an overall positive effect on the generation of new ideas. In addition, our work intends to contribute to the economic-history literature analysing the innovation performance of Italy in the decades following its unification (Barbiellini Amidei et al., 2013; Toninelli and Vasta, 2014; Nuvolari and Vasta 2015, 2017; Nuvolari et al., 2018; Domini, 2023).

The institutional shock we analyse was the outcome of the second war of independence (1859), fought by Piedmont (formally the Kingdom of Sardinia) and France against Austria, with the intent to take from the latter the Lombardo-Venetian Kingdom, broadly corresponding to the present-day regions of Lombardy, Veneto, and Friuli. Austria first entered Piedmont on 27 April 1859, but was pushed back after the arrival of the allied French army. Two months later, the Franco-Piedmontese achieved a decisive victory in Solferino, close to the border between Lombardy and Veneto, after which Napoleon III unexpectedly signed an armistice with Austria. This assigned Lombardy to France, which then transferred it to Piedmont in exchange for Savoy and Nice. This outcome fell short of Piedmont's ambitions, which included also the Eastern part of the kingdom --- so much that prime minister Camillo Benso, count of Cavour, resigned. Veneto only joined what had meanwhile become the Kingdom of Italy (1861) seven years later, in October 1866, after a third war of independence was fought.

We see this as a quasi-random assignment, as the establishment of a new border between Lombardy and Veneto was not anticipated and depended on military and political contingencies. We further argue that Lombardy was ``treated'' with more liberal institutions, as it joined Piedmont, while Veneto remained under repressive Austria. Piedmont was the only Italian state not to repeal the constitution granted in 1848, known as \textit{Statuto Albertino}. This was quite an advanced liberal constitution,  for that time, establishing an elective lower chamber (while the upper one was appointed), and enshrining the principles of equality before the law, individual liberty, freedom of press, and inviolability of private property.
Meanwhile, Austria had a troubled constitutional history: the constitution granted in 1848 was first amended and eventually withdrawn in 1851. A constitution was only re-introduced after the loss of Lombardy, namely the October Diploma (1860), which was soon replaced by the February Patent (1861-1865) and eventually by the December Constitution (1867-1918). Liberal newspaper \textit{Wanderer} (1864) described the February Patent as  ``a constitution without freedom of association, without jury courts, without freedom of press, without equality of confessional rights, lacking a reform of justice and administration.”

While a more advanced constitution may have favoured innovation by establishing stronger individual and property rights, it is important to acknowledge that it was not the only dimension of the ``treatment'' Lombardy was subject to. One that might directly affect Lombardy's innovative performance was a change in the patent system. However, the Piedmontese and Austrian patent systems were similar, as both had been recently reformed (in 1855 and 1852, respectively) along the lines of France's 1844 patent law, establishing a registration system with no novelty examination. Patents were somewhat cheaper on the Italian side, although the difference in fees depended on the duration requested and ranged from zero for a five-year patent to -28.6\% for the maximum duration, 15 years, based on fees reported by Tolhausen (1868). 
Other changes potentially relevant to innovation outcomes were those in market potential (Missiaia, 2016) %-- although this may be expected to be detrimental in the short run, as Lombardy lost access to long-integrated Veneto, as well as to the large Austrian empire, while initially gaining against innovation to  
and in the cultural homogeneity of the new country (Ertug et al., 2021; Mokyr, 2024). %introduces the concept of homophily preference, indicating that similar social preferences can drive collaborative innovation

We measure innovation by means of patent data and international exhibition data. As for the former, we sum Austrian and Piedmontese (Italian, after 1861) patents, to account for the diversion of patenting activity, as Lombardy moved from one state to the other. While patents are a standard measure of innovation, especially in historical settings (Streb, 2023), they suffer from well-known shortcomings, namely they capture invention rather than innovation, and not all inventions are patented, with the propensity to patent widely varying across sectors (Griliches 1990; Nagaoka et al. 2010). As an alternative measure, Moser (2005) introduced the products displayed at international exhibitions. Domini (2019) subsequently argued that these may suffer from opposite drawbacks, compared to patents, as they may represent not necessarily innovative commercialised products. On the bright side, however, exhibits are many more than patents: in 1867, there were 3,841 Italian items on display at the Paris universal exhibition, \textit{vis-à-vis} 164 patents granted to residents in Italy. We use data from the universal exhibitions taking place in Paris in 1855, 1867, 1878, 1889, and 1900. %Notice that the lower frequency of these events implies that this source is rather suited for long-run analysis. 

The empirical strategy makes use of two designs. The first is a cross-sectional regression discontinuity (RD):

\begin{equation*}
    Y_{i} = \alpha + \beta D_i + \gamma Dist_i + \epsilon_i 
\end{equation*}

\noindent where $ D_i $ is a dummy equal to 1 if municipality $ i $ is located in the territory annexed by Piedmont after the war in 1859, $ Dist_i $ is distance from the new border, and the sample includes observations within a certain distance from the border. The second design is a difference-in-differences (DD):

\begin{equation*}
    Y_{it} = \alpha_i + \beta_1 \text{Post}_t + \beta_2 D_i \cdot \text{Post}_t + \epsilon_{it}
\end{equation*}

\noindent where $ Post_t $ indicates years after 1859. The dependent variable $ Y_i $ is, alternatively, the sum of Austrian and Piedmontese (Italian) patents, the count of exhibits, or the top complexity (Hidalgo and Hausmann, 2009) of the products exhibited by municipality $ i $, based on product complexity estimates by Domini (2022), which can be interpreted as a measure of $ i $'s capabilities.

Our preliminary results, both from RD and DD, show a positive short-run effect of Lombardy's annexation on patents, though not on exhibits. It is important to mention that event studies do not reveal pre-trends, which lends credibility to the parallel trends assumption, underpinning the DD design. Planned next steps include a regression with exhibit-based complexity as a dependent variable; a long-run analysis, employing both patent and exhibition data up to the turn of the century; and controlling for co-occurring such as that in market potential.  

